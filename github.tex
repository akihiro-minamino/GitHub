\documentclass[11pt, oneside]{article}   	% use "amsart" instead of "article" for AMSLaTeX format
\usepackage{geometry}                		% See geometry.pdf to learn the layout options. There are lots.
\geometry{letterpaper}                   		% ... or a4paper or a5paper or ... 
%\geometry{landscape}                		% Activate for rotated page geometry
%\usepackage[parfill]{parskip}    		% Activate to begin paragraphs with an empty line rather than an indent
\usepackage{graphicx}				% Use pdf, png, jpg, or eps§ with pdflatex; use eps in DVI mode
								% TeX will automatically convert eps --> pdf in pdflatex		
\usepackage{amssymb}

%SetFonts

%SetFonts


\title{GitHub}
\author{Akihiro Minamino}
%\date{}							% Activate to display a given date or no date

\begin{document}
\maketitle
\section{GitHubとは}
GitHubは、コードを共有するための場所を提供しているGitリポジトリのホスティングサービスである。\\

\section{MacへのGitのインストールと初期設定}
Homebrewを用いてインストールするのが簡単である。\\
\verb|$ brew install git|\\
インストールされてgitのversionは、以下で確認できる。\\
\verb|$ git --version|\\


次にGitで利用する名前とメールアドレスを設定する。\\
\verb|$ git config --global user.name "Firstname Lastname"|\\
\verb|$ git config --global user.email "your_email@example.com"|\\
さらにコマンド出力を読みやすくする。
\verb|$ git config --global color.ui auto|\\
上記の内容は、設定ファイル \verb|~/.gitconfig| に書き込まれている。\\

\section{GitHubの利用準備}
GitHubのアカウント作成ページ \verb|https://github.com/join| でアカウントを作成する。\\
この時、「Username」には希望するIDを英数字で入力する。
この「Username」は、公開ページのURLで \verb|https://github.com/Username| として使われる。\\

GitHubでは、作成したリポジトリへのアクセス認証をSSHを利用した公開鍵認証で行う。
まずは、Macで、以下のようにSSH Keyを作成する。\\
\verb|ssh -keygen -t rsa -C "your_email@example.com"|\\
passphraseは「なし」=「何も入力しないでEnter」でよい。\\


この結果、\verb|~/.ssh/| に、\verb|id_rsa|という秘密鍵ファイルと、\verb|id_rsa.pub|という公開鍵ファイルが作成される。\\

GitHubのwebページの右上のアカウント設定ボラン(Account Settings)を押し、「SSH Key」のメニューを選択する。
\begin {itemize}
\item Titleに適当な鍵の名前(例えば「MacBookAirKey」など)を入力する。
\item Keyには、\verb|id_rsa.pub|の内容をコピーして貼り付ける。
\item 公開鍵の登録に成功すると、登録したメールアドレスに公開鍵登録完了のメールが届く。
\end{itemize}
動作確認を以下のとおり行う。\\
\verb|$ ssh -T git@github.com|\\
「Are you sure you want to continue connecting (yes/no)?」には、yesと入力する。\\
次のように表示されれば成功。
\verb|Hi Username! You've successfully authenticated, but GItHub does not provide shell access.|\\


\section{GitHubの使い方}




\verb||\\




\verb||










\end{document}  