\documentclass[11pt, oneside]{article}   	% use "amsart" instead of "article" for AMSLaTeX format
\usepackage{geometry}                		% See geometry.pdf to learn the layout options. There are lots.
\usepackage{listings, jlisting}
\renewcommand{\lstlistingname}{リスト}
\lstset{
%language = R,
%language = C++,   
breaklines = true,
numbers = left,
frame = tbrl,
tabsize = 4,
basicstyle =\ttfamily,
captionpos = t
}
\geometry{letterpaper}                   		% ... or a4paper or a5paper or ... 
%\geometry{landscape}                		% Activate for rotated page geometry
%\usepackage[parfill]{parskip}    		% Activate to begin paragraphs with an empty line rather than an indent
\usepackage{graphicx}				% Use pdf, png, jpg, or eps§ with pdflatex; use eps in DVI mode
								% TeX will automatically convert eps --> pdf in pdflatex		
\usepackage{amssymb}

%SetFonts

%SetFonts


\title{GitHub}
\author{Akihiro Minamino}
%\date{}							% Activate to display a given date or no date

\begin{document}
\maketitle
\section{GitHubとは}
GitHubは、コードを共有するための場所を提供しているGitリポジトリのホスティングサービスである。\\

\section{MacへのGitのインストール}
Homebrewを用いてインストールするのが簡単である。
\begin{lstlisting}
$ brew install git
\end{lstlisting}
インストールされてgitのversionは、以下で確認できる。
\begin{lstlisting}
$ git --version
\end{lstlisting}

\section{Gitの初期設定}
次にGitで利用する名前とメールアドレスを設定する。
\begin{lstlisting}
$ git config --global user.name "Firstname Lastname"
$ git config --global user.email "your_email@example.com"
\end{lstlisting}
さらにコマンド出力を読みやすくする。
\begin{lstlisting}
$ git config --global color.ui auto
\end{lstlisting}
上記の内容は、設定ファイル \verb|~/.gitconfig| に書き込まれている。\\

\section{GitHubの利用準備}
\subsection{GitHubアカウントの作成}
GitHubのアカウント作成ページ \verb|https://github.com/join| でアカウントを作成する。\\
この時、「Username」には希望するIDを英数字で入力する。
この「Username」は、公開ページのURLで \verb|https://github.com/Username| として使われる。\\

\subsection{SSH公開鍵の作成}
GitHubでは、作成したリポジトリへのアクセス認証をSSHを利用した公開鍵認証で行う。
以下のようにSSH Keyを作成する。
\begin{lstlisting}
ssh -keygen -t rsa -C "your_email@example.com"
\end{lstlisting}
(passphraseは「なし」=「何も入力しないでEnter」でよい。)\\
この結果、\verb|~/.ssh/| に、\verb|id_rsa|という秘密鍵ファイルと、\verb|id_rsa.pub|という公開鍵ファイルが作成される。\\

\subsection{GitHubへのSSH公開鍵の登録}
GitHubのwebページの右上のアカウント設定ボタン(Account Settings)を押し、「SSH Key」のメニューを選択する。
\begin {itemize}
\item Titleに適当な鍵の名前(例えば「MacBookAirKey」など)を入力する。
\item Keyには、\verb|id_rsa.pub|の内容をコピーして貼り付ける。
\end{itemize}
公開鍵の登録に成功すると、登録したメールアドレスに公開鍵登録完了のメールが届く。\\ \\
動作確認を以下のとおり行う。
\begin{lstlisting}
$ ssh -T git@github.com
\end{lstlisting}
(「Are you sure you want to continue connecting (yes/no)?」には、yesと入力する。)\\
次のように表示されれば成功。
\begin{lstlisting}
Hi Username! You've successfully authenticated, 
but GitHub does not provide shell access.
\end{lstlisting}


\section{GitHubの使い方}
\subsection{リポジトリの作成}
GitHubのwebページの右上の「+」をクリック後、「New repository」をクリックする。
リポジトリの情報を以下のように入力する。
\begin{itemize}
\item Repository name: リポジトリの名前(例えば Hello-World など)
\item Description: リポジトリの説明(省略可)
\item Public、Private: 公開か非公開かを決める。
\item Initialize this repository with a REAEME: チェックを入れる。
\item Add .gitignore: プルダウンメニューでこのリポジトリで管理する言語やフレームワーク(Python、C++、Texなど)を選択することで、Gitリポジトリでの管理対象外のファイル・ディレクトリを自動で指定してくれる。
\item Add a licence: MITライセンス\footnote{MIT Licence: 1. このソフトウェアを誰でも無償で無制限に扱って良い。ただし、著作権表示および本許諾表示をソフトウェアのすべての複製または重要な部分に記載しなければならない。2. 作者または著作権者は、ソフトウェアに関してなんら責任を負わない。}を選ぶ。
\end{itemize}
作成したリポジトリのURLは以下である。\\
\verb|https://github.com/ユーザー名/リポジトリ名|\\

\subsection{自分のPCにリポジトリのcloneを作成}
作成したリポジトリのcloneを自分のPCに持ってくる。
\begin{lstlisting}
$ git clone git@github.com:ユーザー名/リポジトリ名
$ cd リポジトリ名
\end{lstlisting}

\subsection{自分のPCのリポジトリにコードをcommit}
自分のPCのリポジトリのcloneで、コード( ここでは、hello\_world.py とする)を作成する。
この時点では、hello\_world.py はGitリポジトリに登録されていないので、\verb|git status|では、Untracked filesとして表示される。\\
 \\
次に、git addコマンドで、hello\_world.py をステージ\footnote{コミットする前のファイル状態を記録したインデックスというデータ構造に記録すること}する。
\begin{lstlisting}
git add hello_world.py
\end{lstlisting}
その後、git commitコマンドで、hello\_worldを自分のPCのリポジトリにコミットする。
\begin{lstlisting}
$ git commit -m "コメント"
\end{lstlisting}

\subsection{GitHub側のリポジトリを更新}
git pushで、GitHub側のリポジトリを更新する。
\begin{lstlisting}
$ git push
\end{lstlisting}


\section{GitHubの操作}
\subsection{ワークツリー}
.gitディレクトリを持つディレクトリ以下を、リポジトリに付随したワークツリーと呼ぶ。
開発者は、ワークツリーでファイルの編集などを行う。
その後、ファイルの編集履歴などを.gitディレクトリに登録して管理する。\\

\subsection{git status}
git statusコマンドは、Gitリポジトリの状態を調べるのに利用する。
\begin{lstlisting}
$ git status
\end{lstlisting}

\subsection{git add}
Gitリポジトリのワークツリーでファイルを作成しただけでは、Gitリポジトリにファイルは登録されない。このようなファイルは、git statusコマンドで、「Untracked files」と表示される。\\
ファイルをGitリポジトリの管理対象とするには、git addコマンドで、ステージ領域\footnote{ステージ領域はインデックスとも呼ばれる。}にファイルを登録する。
ステージ領域は、コミットをする前の一時領域のことである。
\begin{lstlisting}
$ git add ファイル名
\end{lstlisting}

\subsection{git commit}
git commitコマンドは、ステージ領域に登録されている時点のファイルを、リポジトリの歴史として記録するコマンドである。
\begin{lstlisting}
$ git commit -m "コメント"
\end{lstlisting}

\subsection{git log}
git logコマンドは、リポジトリにコミットされたログを確認するコマンドである。
誰がいつコミットやマージを行い、どのような差分が発生したかなどを確認できる。
\begin{lstlisting}
$ git log
\end{lstlisting}

指定したディレクトリ、ファイルのみログを表示するには、git logコマンドにディレクトリ名もしくはファイル名を与えればよい。
\begin{lstlisting}
$ git log ディレクトリ名(ファイル名)
\end{lstlisting}

各コミットでのファイルの差分を表示したい場合は、git log -pと-pのオプションをつける。
\begin{lstlisting}
$ git log -p
\end{lstlisting}

\subsection{.gitignore}
gitの管理下に置きたくないファイルは、.gitignoreに記入すればよい。\\
 \\
.gitignoreの書き方の例\\
ファイル名
\begin{lstlisting}
sample.html
\end{lstlisting}
コメント: \#で始まる行はコメント
\begin{lstlisting}
# コメントです
\end{lstlisting}
ディレクトリ: ディレクトリ名は、末尾に\/を付けて指定
\begin{lstlisting}
work/
\end{lstlisting}




\end{document}  